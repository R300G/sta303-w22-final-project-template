%%
% Copyright (c) 2017 - 2021, Pascal Wagler;
% Copyright (c) 2014 - 2021, John MacFarlane
%
% All rights reserved.
%
% Redistribution and use in source and binary forms, with or without
% modification, are permitted provided that the following conditions
% are met:
  %
% - Redistributions of source code must retain the above copyright
% notice, this list of conditions and the following disclaimer.
%
% - Redistributions in binary form must reproduce the above copyright
% notice, this list of conditions and the following disclaimer in the
% documentation and/or other materials provided with the distribution.
%
% - Neither the name of John MacFarlane nor the names of other
% contributors may be used to endorse or promote products derived
% from this software without specific prior written permission.
%
% THIS SOFTWARE IS PROVIDED BY THE COPYRIGHT HOLDERS AND CONTRIBUTORS
% "AS IS" AND ANY EXPRESS OR IMPLIED WARRANTIES, INCLUDING, BUT NOT
% LIMITED TO, THE IMPLIED WARRANTIES OF MERCHANTABILITY AND FITNESS
% FOR A PARTICULAR PURPOSE ARE DISCLAIMED. IN NO EVENT SHALL THE
% COPYRIGHT OWNER OR CONTRIBUTORS BE LIABLE FOR ANY DIRECT, INDIRECT,
% INCIDENTAL, SPECIAL, EXEMPLARY, OR CONSEQUENTIAL DAMAGES (INCLUDING,
                                                            % BUT NOT LIMITED TO, PROCUREMENT OF SUBSTITUTE GOODS OR SERVICES;
                                                            % LOSS OF USE, DATA, OR PROFITS; OR BUSINESS INTERRUPTION) HOWEVER
% CAUSED AND ON ANY THEORY OF LIABILITY, WHETHER IN CONTRACT, STRICT
% LIABILITY, OR TORT (INCLUDING NEGLIGENCE OR OTHERWISE) ARISING IN
% ANY WAY OUT OF THE USE OF THIS SOFTWARE, EVEN IF ADVISED OF THE
% POSSIBILITY OF SUCH DAMAGE.
%%

  %%
  % This is the Eisvogel pandoc LaTeX template.
%
% For usage information and examples visit the official GitHub page:
  % https://github.com/Wandmalfarbe/pandoc-latex-template
%%

  % Options for packages loaded elsewhere
\PassOptionsToPackage{unicode}{hyperref}
\PassOptionsToPackage{hyphens}{url}
\PassOptionsToPackage{dvipsnames,svgnames*,x11names*,table}{xcolor}
    %
\documentclass[
          english,
          paper=a4,
              ,captions=tableheading
  ]{scrartcl}
    \usepackage{amsmath,amssymb}
    \usepackage{lmodern}
      \usepackage{setspace}
\setstretch{1.2}
  \usepackage{ifxetex,ifluatex}
\ifnum 0\ifxetex 1\fi\ifluatex 1\fi=0 % if pdftex
\usepackage[T1]{fontenc}
\usepackage[utf8]{inputenc}
\usepackage{textcomp} % provide euro and other symbols
\else % if luatex or xetex
  \usepackage{unicode-math}
  \defaultfontfeatures{Scale=MatchLowercase}
\defaultfontfeatures[\rmfamily]{Ligatures=TeX,Scale=1}
                \fi
  % Use upquote if available, for straight quotes in verbatim environments
\IfFileExists{upquote.sty}{\usepackage{upquote}}{}
\IfFileExists{microtype.sty}{% use microtype if available
  \usepackage[]{microtype}
  \UseMicrotypeSet[protrusion]{basicmath} % disable protrusion for tt fonts
}{}
  \makeatletter
\@ifundefined{KOMAClassName}{% if non-KOMA class
  \IfFileExists{parskip.sty}{%
    \usepackage{parskip}
  }{% else
    \setlength{\parindent}{0pt}
    \setlength{\parskip}{6pt plus 2pt minus 1pt}}
}{% if KOMA class
  \KOMAoptions{parskip=half}}
\makeatother
    \usepackage{xcolor}
\definecolor{default-linkcolor}{HTML}{A50000}
\definecolor{default-filecolor}{HTML}{A50000}
\definecolor{default-citecolor}{HTML}{4077C0}
\definecolor{default-urlcolor}{HTML}{4077C0}
\IfFileExists{xurl.sty}{\usepackage{xurl}}{} % add URL line breaks if available
  \IfFileExists{bookmark.sty}{\usepackage{bookmark}}{\usepackage{hyperref}}
\hypersetup{
      pdftitle={New and Affordable `Active' and `Advance' Mingar Wearable Lines Turning Heads\ldots{} Maybe Not All the Ones They Wanted},
                pdfauthor={Report prepared for MINGAR by Simplicity},
                      pdflang={en},
                                                            hidelinks,
                              breaklinks=true,
                pdfcreator={LaTeX via pandoc with the Eisvogel template}}
\urlstyle{same} % disable monospaced font for URLs
      \usepackage[margin=2.5cm,includehead=true,includefoot=true,centering,]{geometry}
                            \usepackage{longtable,booktabs,array}
      \usepackage{calc} % for calculating minipage widths
      % Correct order of tables after \paragraph or \subparagraph
  \usepackage{etoolbox}
  \makeatletter
  \patchcmd\longtable{\par}{\if@noskipsec\mbox{}\fi\par}{}{}
  \makeatother
  % Allow footnotes in longtable head/foot
  \IfFileExists{footnotehyper.sty}{\usepackage{footnotehyper}}{\usepackage{footnote}}
  \makesavenoteenv{longtable}
          % add backlinks to footnote references, cf. https://tex.stackexchange.com/questions/302266/make-footnote-clickable-both-ways
      \usepackage{footnotebackref}
          \usepackage{graphicx}
  \makeatletter
  \def\maxwidth{\ifdim\Gin@nat@width>\linewidth\linewidth\else\Gin@nat@width\fi}
  \def\maxheight{\ifdim\Gin@nat@height>\textheight\textheight\else\Gin@nat@height\fi}
  \makeatother
  % Scale images if necessary, so that they will not overflow the page
  % margins by default, and it is still possible to overwrite the defaults
  % using explicit options in \includegraphics[width, height, ...]{}
  \setkeys{Gin}{width=\maxwidth,height=\maxheight,keepaspectratio}
  % Set default figure placement to htbp
  \makeatletter
  \def\fps@figure{htbp}
  \makeatother
                      \setlength{\emergencystretch}{3em} % prevent overfull lines
      \providecommand{\tightlist}{%
        \setlength{\itemsep}{0pt}\setlength{\parskip}{0pt}}
              \setcounter{secnumdepth}{-\maxdimen} % remove section numbering
                                              
            % Make use of float-package and set default placement for figures to H.
          % The option H means 'PUT IT HERE' (as  opposed to the standard h option which means 'You may put it here if you like').
          \usepackage{float}
          \floatplacement{figure}{H}

                      \usepackage{booktabs}
                      \usepackage{longtable}
                      \usepackage{array}
                      \usepackage{multirow}
                      \usepackage{wrapfig}
                      \usepackage{float}
                      \usepackage{colortbl}
                      \usepackage{pdflscape}
                      \usepackage{tabu}
                      \usepackage{threeparttable}
                      \usepackage{threeparttablex}
                      \usepackage[normalem]{ulem}
                      \usepackage{makecell}
                      \usepackage{xcolor}
                                    \ifxetex
                      % See issue https://github.com/reutenauer/polyglossia/issues/127
          \renewcommand*\familydefault{\sfdefault}
                      % Load polyglossia as late as possible: uses bidi with RTL langages (e.g. Hebrew, Arabic)
          \usepackage{polyglossia}
          \setmainlanguage[]{english}
                      \else
              \usepackage[main=english]{babel}
          % get rid of language-specific shorthands (see #6817):
                                                     \let\LanguageShortHands\languageshorthands
                                                     \def\languageshorthands#1{}
                                                                                                            \fi
                                                                                                            \ifluatex
                                                     \usepackage{selnolig}  % disable illegal ligatures
                                                     \fi
                                                                                                                                                                                                                                      
\title{New and Affordable `Active' and `Advance' Mingar Wearable Lines
Turning Heads\ldots{} Maybe Not All the Ones They Wanted}
\usepackage{etoolbox}
\makeatletter
\providecommand{\subtitle}[1]{% add subtitle to \maketitle
  \apptocmd{\@title}{\par {\large #1 \par}}{}{}
}
\makeatother
\subtitle{Who are these new customers? And do these racist complaints
hold any merit?}
\author{Report prepared for MINGAR by Simplicity}
\date{2022-04-11}



%%
%% added
%%

%
% language specification
%
% If no language is specified, use English as the default main document language.
%



%
% for the background color of the title page
%
\usepackage{pagecolor}
\usepackage{afterpage}
\usepackage[margin=2.5cm,includehead=true,includefoot=true,centering]{geometry}

%
% break urls
%
\PassOptionsToPackage{hyphens}{url}

%
% When using babel or polyglossia with biblatex, loading csquotes is recommended
% to ensure that quoted texts are typeset according to the rules of your main language.
%
\usepackage{csquotes}

%
% captions
%
\definecolor{caption-color}{HTML}{777777}
\usepackage[font={stretch=1.2}, textfont={color=caption-color}, position=top, skip=4mm, labelfont=bf, singlelinecheck=false, justification=raggedright]{caption}
\setcapindent{0em}

%
% blockquote
%
\definecolor{blockquote-border}{RGB}{221,221,221}
\definecolor{blockquote-text}{RGB}{119,119,119}
\usepackage{mdframed}
\newmdenv[rightline=false,bottomline=false,topline=false,linewidth=3pt,linecolor=blockquote-border,skipabove=\parskip]{customblockquote}
\renewenvironment{quote}{\begin{customblockquote}\list{}{\rightmargin=0em\leftmargin=0em}%
\item\relax\color{blockquote-text}\ignorespaces}{\unskip\unskip\endlist\end{customblockquote}}


%
% heading color
%
\definecolor{heading-color}{RGB}{40,40,40}
\addtokomafont{section}{\color{heading-color}}
% When using the classes report, scrreprt, book,
% scrbook or memoir, uncomment the following line.
%\addtokomafont{chapter}{\color{heading-color}}

%
% variables for title, author and date
%
\usepackage{titling}
\title{New and Affordable `Active' and `Advance' Mingar Wearable Lines
Turning Heads\ldots{} Maybe Not All the Ones They Wanted}
\author{Report prepared for MINGAR by Simplicity}
\date{2022-04-11}

%
% tables
%

\definecolor{table-row-color}{HTML}{F5F5F5}
\definecolor{table-rule-color}{HTML}{999999}

%\arrayrulecolor{black!40}
\arrayrulecolor{table-rule-color}     % color of \toprule, \midrule, \bottomrule
\setlength\heavyrulewidth{0.3ex}      % thickness of \toprule, \bottomrule
\renewcommand{\arraystretch}{1.3}     % spacing (padding)


%
% remove paragraph indention
%
\setlength{\parindent}{0pt}
\setlength{\parskip}{6pt plus 2pt minus 1pt}
\setlength{\emergencystretch}{3em}  % prevent overfull lines

%
%
% Listings
%
%


%
% header and footer
%
\usepackage[headsepline,footsepline]{scrlayer-scrpage}

\newpairofpagestyles{eisvogel-header-footer}{
  \clearpairofpagestyles
  \ihead[2022-04-11]{New and Affordable `Active' and `Advance' Mingar
Wearable Lines Turning Heads\ldots{} Maybe Not All the Ones They Wanted}
  \chead[]{}
  \ohead[New and Affordable `Active' and `Advance' Mingar Wearable Lines
Turning Heads\ldots{} Maybe Not All the Ones They Wanted]{2022-04-11}
  \ifoot[\thepage]{Report prepared for MINGAR by Simplicity}
  \cfoot[]{}
  \ofoot[Report prepared for MINGAR by Simplicity]{\thepage}
  \addtokomafont{pageheadfoot}{\upshape}
}
\pagestyle{eisvogel-header-footer}

%%
%% end added
%%

\begin{document}

%%
%% begin titlepage
%%
\begin{titlepage}
\newgeometry{left=6cm}
\definecolor{titlepage-color}{HTML}{6C3082}
\newpagecolor{titlepage-color}\afterpage{\restorepagecolor}
\newcommand{\colorRule}[3][black]{\textcolor[HTML]{#1}{\rule{#2}{#3}}}
\begin{flushleft}
\noindent
\\[-1em]
\color[HTML]{FFFFFF}
\makebox[0pt][l]{\colorRule[FFFFFF]{1.3\textwidth}{2pt}}
\par
\noindent

{
  \setstretch{1.4}
  \vfill
  \noindent {\huge \textbf{\textsf{New and Affordable `Active' and
`Advance' Mingar Wearable Lines Turning Heads\ldots{} Maybe Not All the
Ones They Wanted}}}
    \vskip 1em
  {\Large \textsf{Who are these new customers? And do these racist
complaints hold any merit?}}
    \vskip 2em
  \noindent {\Large \textsf{Report prepared for MINGAR by Simplicity}}
  \vfill
}


\textsf{2022-04-11}
\end{flushleft}
\end{titlepage}
\restoregeometry

%%
%% end titlepage
%%



{
\setcounter{tocdepth}{2}
\tableofcontents
}
\newpage

\hypertarget{executive-summary}{%
\section{Executive summary}\label{executive-summary}}

Our consultants at Simplicity, a company that makes consulting simple
for its clients, will be tackling Mingar's most pressing questions today
regarding their high-end fitness wearing tracking wearable devices. More
specifically, we will be exploring their newest and most affordable
lines of wearables, ``Active'' and ``Advance'', to uncover who their
current audience is and what they find so attractive about these
products.

Turns out, it is most likely the affordability. Customers of the new
``Active'' and ``Advance'' lines are between the ages of 60 or older and
30 or younger. One thing these age groups have are thier lower income
levels. On average, incomes peak for individuals between 30 to 60.
Moreover, household median income of the customer by neighbourhood also
agrees with the finding that lower income earners prefer ``Active'' and
``Advance'' lines, while high income earners prefer others.

Secondly, there has been a trend of complaints that Mingar wearables are
``racist'', as they were reported to perform poorly for darker
complexions compared to thier lighter counter parts. Our findings agreed
with those complaints. On the spectrum of skin tones, we found that the
number of quality flag raised during sleep increased as the complexion
deepened. This may be due to some data recording quality issue or sensor
incompetencies.

This report will utilize our strengths to uncover clever manipulations
and wrangling of the data to best answer these questions with the
limited information available.

\newpage

\hypertarget{technical-report}{%
\section{Technical report}\label{technical-report}}

\hypertarget{introduction}{%
\subsection{Introduction}\label{introduction}}

In this report, Simplicity will tackle Mingar's most pressing questions
to date, i.e.~who are Mingar's customers of their newer and more
cost-efficient wearable lines? And do the currently trending damaging
claims of ``racist'' Mingar wearables hold any weight? Read on to find
out.

Our company has gone through the exhaustive work of selecting the 2 best
fitting models of the data that best answer each question by considering
any predictors that may influence our variable of interest, within the
realm of common sense and reason backed by statistical tests, handy
plots and colourful illustrations. Due to racial and ethnic privacy
issues that prevent Mingar from collecting such data, we have rolled up
our sleeves, wrangled the data and gotten creative with our approach to
solving our client's questions. And in the spirit of transparency, we
will explain any limitations that we ran into in the process.

\hypertarget{research-questions}{%
\subsubsection{Research questions}\label{research-questions}}

Our report aims to address the 2 questions that Mingar has asked of us.
More specifically, Question 1 targets primarily who Mingar's new
customers of the affordable lines of wearables, ``Active'' and
``Advance'' are. We aim to highlight the specific significant factors
that do and do not identify such customers.

Question 2 asks to find any device incompetencies found during sleep
score flag records due to user skin color likely stemming from sensor
quality issue or other data quality issue to determine whether there is
racial bias existing within Mingar's wearables.

\hypertarget{who-are-the-active-and-advance-line-wearables-customers}{%
\subsection{Who are the ``Active'' and ``Advance'' Line Wearables
Customers?}\label{who-are-the-active-and-advance-line-wearables-customers}}

The data used for the purpose of this research question is based on the
joint data of customer device, customer, and device data provided from
the client Mingar. Then we further strengthen the data information by
including living areas sorted by postal code, median income of areas,
and population of the areas. Based on the existing variables, column age
was created using Date of Birth information from the data. Column emoji
color was also created based on the variable emoji modifier in the data.
Users preferred emoji skin color can be used as an rough indicator of
their skin color, if user uses no emoji modifier, the variable emoji
color will default to yellow. However, yellow has zero indications on a
user's potential skin color as default emoji face color without any
alteration is yellow. Yellow can be treated as unkown potential skin
color.

Let us begin by discussing the process surrounding the model selection
for this question. We know that we want our response variable to be
binomial, where 0 represents a user that does not own an active or
advance wearable and 1 represents a user that does own anactive or
advance wearable. Thus the underlying distribution of the generalized
linear model(GLM) is binomial.

In the first stages of our analysis, we began by looking at the
relationship different variables may have had on our response variable.

\includegraphics{sta303-final-project_files/figure-latex/1-1.pdf} The
density distribution figure above illustrates that users less than 30
and older than 60 own more active or advance line wearables than users
of the same age group for other lines. This leads us to wonder whether
age is a valid model predictor.

\includegraphics{sta303-final-project_files/figure-latex/2-1.pdf} The
next density distribution showcases the relationship between household
median income for active or advance line users versus other.The figure
above shows that customers with a household median income of
approximately \$70,000 or less own more of the active or advance line
wearables than otherwise.

Once again suggesting that a potential product for determining active or
advance customers from others is household median income.

\includegraphics{sta303-final-project_files/figure-latex/3-1.pdf}
Afterwards, we observe our third density distribution figure for this
question, this time modeling the population variable, which represents
the population of wearables users within a postal code/neighbourhood,
against our response variable. However, no discernible pattern or
findings can be drawn from its odd shape.

\includegraphics{sta303-final-project_files/figure-latex/4-1.pdf} My
partner and I were interested in whether the sex of a user differed
based on the different lines of wearables. As you can see, the
percentage of female, male, intersex and unknown barely altered when
coming active or advance users against all others. We hypothesis at this
point that sex will not be a predictor for our model.

\includegraphics{sta303-final-project_files/figure-latex/5-1.pdf} We
display another graph resembling the one above except this one compares
emoji skin tone against lines of wearables. Similarly, no difference
between active or advance users and other line wearable users in their
percentage distribution

\hypertarget{model-selection}{%
\subsubsection{Model Selection}\label{model-selection}}

To identify the target audiences for Mingar's new budget line up of
devices Active and Advance, Simplicity decides to a binomial generalized
linear model. The reason to select such model is due to the nature of
the variable we are to study, active advance users. It consists of 1 and
0 where 1 represents user uses active or advance devices, and 0
otherwise. We want to determine the demographic backgrounds of active
and advance users by using active advance users as outcome and various
demographic background as the predictors. And a generalized linear model
is a perfect fit for such task. Through multiple nested likelihood ratio
tests, testing on background variables like median income, age, sex,
location, and potential skin colors, The most appropriate model is given
by:

\(ActiveAdvanceUsers \sim Binomial(N,p)\)

\(log(\frac{p}{1-p})={\beta_0} +{\beta_1}MedianIncome+{\beta_2}Age+{\beta_3}Population\)

Where \(N\) represents the number of Device Users, and \(p\) represents
the probability of a device user uses Active or Advance given the
information on median income and population.

\begin{longtable}[]{@{}ccccccc@{}}
\caption{Summary Table of The Generalized Linear Model}\tabularnewline
\toprule
term & estimate & std.error & statistic & p.value & conf.low &
conf.high \\
\midrule
\endfirsthead
\toprule
term & estimate & std.error & statistic & p.value & conf.low &
conf.high \\
\midrule
\endhead
(Intercept) & 0.817 & 0.063 & 12.934 & 0 & 0.693 & 0.941 \\
scales::rescale(hhld\_median\_inc) & -3.318 & 0.192 & -17.261 & 0 &
-3.695 & -2.942 \\
scales::rescale(age) & 0.377 & 0.075 & 5.041 & 0 & 0.230 & 0.523 \\
Population & 0.000 & 0.000 & -4.185 & 0 & 0.000 & 0.000 \\
\bottomrule
\end{longtable}

After conducting likelihood ratio tests, we found that the best suitable
model for our binary response variable selected household median income,
age and population as significant predictors, not surprisingly.
Furthermore, we see that there is a negative relationship between income
and our active and advance line users, which leads me to interpret that
lower income earners prefer the active and advance wearable lines, while
higher income earners prefer other wearables. Further to our findings
from above regarding age, we do see that it does play a significant role
in characterizing our active and advance line customers. The
significance of the population variable leaves lots of rooms for
questions as it is unclear as to how or why the number of people in a
neighbourhood would play a role in determining a consumer's choice of
active or advance line of wearables versus others. One theory I will
pose is that perhaps the more individuals there are in a neighbourhood,
the more likely those occupants will follow trends and purchase more
afforable wearables.

\hypertarget{model-assumptions}{%
\subsubsection{Model Assumptions}\label{model-assumptions}}

For a generalized linear model to be valid, following assumptions needs
to hold.

\begin{enumerate}
\def\labelenumi{\arabic{enumi}.}
\item
  The model's response, the probability of Active Advance Users among
  the users should be independently distributed, and the errors are also
  assumed to be independent
\item
  The model assumes a linear relationship between the transformed
  response with the predictor variables. Which in this case is the
  \(log(\frac{p}{1-p})\) and \(age, median\ income,\) and
  \(population\). Where \(p\) represents the probability of a device
  user uses Active or Advance given the information on median income and
  population.
\end{enumerate}

\includegraphics{sta303-final-project_files/figure-latex/unnamed-chunk-5-1.pdf}
As seen from the figure, both age and median income have clear linear
relationship with the transformed response, while population fails to
reach the assumptions.

Although Population is significant predictor to the model, due to its
low treatment effect on the probability of active or advance users, and
its failure to reach the model assumptions. The predictor is removed
from the model.

The new model is given by:

\(ActiveAdvanceUsers \sim Binomial(N,p)\)

\(log(\frac{p}{1-p})={\beta_0} +{\beta_1}MedianIncome+{\beta_2}Age\)

Where \(N\) represents the number of Device Users, and \(p\) represents
the probability of a device user uses Active or Advance given the
information on median income and population.

\begin{longtable}[]{@{}ccccccc@{}}
\caption{Summary Table of The Generalized Linear Model}\tabularnewline
\toprule
term & estimate & std.error & statistic & p.value & conf.low &
conf.high \\
\midrule
\endfirsthead
\toprule
term & estimate & std.error & statistic & p.value & conf.low &
conf.high \\
\midrule
\endhead
(Intercept) & 0.647 & 0.048 & 13.450 & 0 & 0.553 & 0.741 \\
scales::rescale(hhld\_median\_inc) & -3.043 & 0.180 & -16.907 & 0 &
-3.396 & -2.691 \\
scales::rescale(age) & 0.376 & 0.075 & 5.042 & 0 & 0.230 & 0.523 \\
\bottomrule
\end{longtable}

The model summary and confidence interval are almost identical to the
previously proposed model, just without population as one of the model's
predictor.

\hypertarget{are-mingar-wearables-racist}{%
\subsection{Are Mingar Wearables
Racist?}\label{are-mingar-wearables-racist}}

The data used to determine the cause of poor sleep score performance of
the devices is based on the previous data used for the first research
question. Then we joint the data with customer sleep data provided by
Mingar. The data allows us to explore and validate whether having darker
skin color actually causes the devices to perform poorly on sleep
functionalities.

The showcase of poor sleep performance of the wearable can be indicated
by two variables, user's sleep duration, and the number of malfunction
red flags that device records over user's sleep session. Thus, one of
the variable will become the output of this study's model.

\begin{figure}
\centering
\includegraphics{sta303-final-project_files/figure-latex/6-1.pdf}
\caption{Relationship of average Sleep Duration and average Flags per
User's Sleep}
\end{figure}

Figure 1 illustrates a positive linear relationships between average
sleep duration and average red flags per user's sleep. This can be
easily interpreted as when sleep duration last longer, wearable devices
have more opportunity to malfunction and flags.

Now we wish to compare both sleep duration and flags to demographic
backgrounds in order to determine further information for the
construction of te model.

\includegraphics{sta303-final-project_files/figure-latex/7-1.pdf} Figure
2 showcases a negative relation between sleep duration and darker skin
color. As user's potential skin color gets darker, the sleep duration is
expected to drop as well, resulting lower sleep qualities that's
possibly due to malfunctions of the wearable.

\begin{figure}
\centering
\includegraphics{sta303-final-project_files/figure-latex/8-1.pdf}
\caption{Distribution of average Flags per User's sleep by preferred
Emoji Skin Tone}
\end{figure}

Figure 3 displayed an extremely positive trend of average flags with
darker skin color. Reading from this chart, potential skin color is
almost certainly a very significant predictor for the determining the
average flags per user's sleep, as the number of flags gets higher for
every darker change of skin tone.

\includegraphics{sta303-final-project_files/figure-latex/9-1.pdf} Figure
4 showcased a negative linear relationship between sleep durations and
age. Although it is possible that age is a influential variable on
Mingar wearable's underperformance on sleep score. It is also very
reasonble to interpret the trend as user tends to have lower sleep
quality and less sleep time when they get older, due to health and many
other reasons.

\begin{figure}
\centering
\includegraphics{sta303-final-project_files/figure-latex/10-1.pdf}
\caption{Relationship of average Flags and Age per User's sleep}
\end{figure}

Unlike figure 4, figure 5 actually almost shows no linear trends btween
average number of flags and age. The distribution of the points are
rather flat, indicating age actually have no influence on the number of
malfunctions and poor performances of the devices. Thus, it is unlikely
that the negative trend displayed in figure 4 is caused by wearable's
underperformance on older users. This indicates flags is a better output
for the model than sleep duration, as number of flags indicated the
number errors caused by the machine, while sleep duration includes
broader interpretations where a clear trend could possibily caused by
other reason than device's fault.

Moving on, the report will only explore the relationship flags with the
rest of the demographics.

\includegraphics{sta303-final-project_files/figure-latex/11-1.pdf} No
differences of average number of flags was displayed between genders.
All three type of genders shares similar distributions of flags with the
same mean. This indicates that sex have no influence on flags.

\begin{figure}
\centering
\includegraphics{sta303-final-project_files/figure-latex/12-1.pdf}
\caption{Relationship of average flags per User's Sleep and User's
Postal Code Population}
\end{figure}

Similarly to sex, population also shows no clear trends with the number
of average flags.

\hypertarget{model-selection-1}{%
\subsubsection{Model Selection}\label{model-selection-1}}

To identify the root cause for Mingar's some device performing poorly on
sleep scores, Simplicity decides to use a Poisson generalized linear
mixed model. The first reason to select such a model is due to the
nature of the variable we are studying. Flags is a count variable that
records the number of red flags of device malfunction over the period of
a user's sleep, thus falls under Poisson distribution. Due to the
collection method of the data, the data records large amount of
observations of sleep from each users, therefore it is appropriate to
include customer id as random effect, which makes the model a linear
mixed model. From previous data discovery, we learned that flags and
sleep duration have a positive linear relation, more flags can
potentially appear as sleep lasts longer. To standardized number of
flags within the same period of time, log sleep duration is selected to
be the offset of the model.

Lastly, we want to determine whether skin color and other demographic
backgrounds influences the performance of the devices,

Through multiple nested likelihood ratio tests, testing on background
variables median income, age, sex, location, and potential skin colors.

The most appropriate model is given by:

\(log(flags)= log(duration)+{\beta_1}emoji\_colorYellow+{\beta_2}emoji\_colorLight+{\beta_3}emoji\_colorMedium Light+ {\beta_4}emoji\_colorMedium+{\beta_5}emoji\_colorMedium Dark+{\beta_6}emoji\_colorDark+ U_i\)

Where \(U_i\) represents random effect for \(Customer_i\). Because only
dummy variables were used as predictors, the report decides not to use
intercept for this model, and there will be no reference level.

Across all demographic backgrounds, Only potential skin colors are
significant to the Poisson regression model.This validates the complaint
from customers that devices are performing poorly for users with darker
skin, particularly with respect to sleep scores.

\begin{longtable}[]{@{}cccccccc@{}}
\caption{Summary Table of The Poisson Generalized Linear Mixed
Model}\tabularnewline
\toprule
effect & term & estimate & std.error & statistic & p.value & conf.low &
conf.high \\
\midrule
\endfirsthead
\toprule
effect & term & estimate & std.error & statistic & p.value & conf.low &
conf.high \\
\midrule
\endhead
fixed & emoji\_colorDark & 0.033 & 0 & -463.509 & 0 & 0.033 & 0.034 \\
fixed & emoji\_colorLight & 0.003 & 0 & -365.286 & 0 & 0.003 & 0.003 \\
fixed & emoji\_colorMedium & 0.010 & 0 & -435.055 & 0 & 0.010 & 0.010 \\
fixed & emoji\_colorMedium-Dark & 0.020 & 0 & -474.205 & 0 & 0.020 &
0.021 \\
fixed & emoji\_colorMedium-Light & 0.007 & 0 & -416.901 & 0 & 0.006 &
0.007 \\
fixed & emoji\_colorYellow & 0.007 & 0 & -536.011 & 0 & 0.006 & 0.007 \\
\bottomrule
\end{longtable}

Examining the table of summary, all skin colors are extremely
significant to the estimation of the model. As expected Dark have high
positive influence on the number of flags, with an increase of 0.033
flags per minute in sleep. All skin colors coefficents are captured
within the confidence interval as well.

\hypertarget{model-assumptions-1}{%
\subsubsection{Model Assumptions}\label{model-assumptions-1}}

For a Poisson generalized linear mixed model to be valid, following
assumptions needs to hold.

\begin{enumerate}
\def\labelenumi{\arabic{enumi}.}
\item
  The model's response needs to be under Poisson distribution. Flags is
  a count variable that records the number of red flags over the period
  of user's sleep time, thus falls under Poisson distribution.
\item
  Random effect, which in this case are the customers should be normally
  distributed.
\item
  The customer's random effects errors and residual errors within
  preferred emoji skin should have constant variance.
\item
  The model assumes a that the transformed response must have a linear
  function with predictor variables. Which in this case is
  \(log(Flags)\) and \(emoji \\color\).
\end{enumerate}

In conclusion, devices are definitely performing poorly for users with
darker skin, particularly with respect to sleep scores. \#\# Discussion

After selecting the most appropriate model for each question, we can now
answer the 2 questions at hand.

Who Are the ``Active'' and ``Advance'' Line Wearables Customers? First,
we have to answer what the distinct characteristics of the ``Active''
and ``Advance'' customers are. Through the final selected generalized
linear model, we see the most significant character traits to be income,
age and number of residents in the same neighbourhood. Furthermore, we
see that there is a negative relationship between the active and advance
customers binary response variable and income, which reasons that
wealthier individuals buy less of these 2 lines, meanwhile lower income
individuals buy more of them. In one of our plots above, we see that
customers less than 20 and over 60 buy more of the active and advance
lines than the others. This suggests that once again adolescents or
young adults and seniors are more the target audience for these lines.
This is congruent to our prior conclusions regarding income,
i.e.~individuals in those age groups are generally regarded as having
lower income looking for more affordable products. As for the last
characteristic, population within a neighbourhood, conclusions are hard
to draw as the relationship is unclear, though notably it is negative.

Are Mingar Wearables ``Racist''? In short, yes. Let's talk about
it\ldots{} Through our generalized linear mixed model, it was evident
that the number of flags, which are quality issues that occur during
sleep, related to sensor error or other data quality issues, increased
for darker skin tones of overall the emoji modifiers; in exactly the
order of Light, Medium Light, Medium, Medium Dark to Dark (least flags
to most flags respectively). Moreover, out of the six models we
considered, the most suitable model found the emoji modifier predictor
to be the only significant one for the flag response variable, leaving
us with the belief that the only consequential factor of these flags is
the individual's skin tone. Our recommendation is that further tests on
the hardware of the wearables should be conducted to determine the exact
cause behind the skin tone sensitivity and lack of quality as a result.
\#\#\# Strengths and limitations

Simplicity believes in transparency and as such, we find it important to
disclose the limitations of our research and findings in order for our
clients to feel confident with what they are receiving. Since we do not
have information on individual income, we had to improvise. We cleverly
pulled income information based on postal code from the postal code
conversion file (2016 census geography, August 2021 postal codes)
provided to us as University of Toronto students. Thus, we only have a
general sense of the wealth of a customer based on the neighbourhood
they live in. It may not be exactly accurate per consumer considering
that high income individuals may reside in low income neighbourhoods and
the opposite is true too; however we believe that our analysis provides
insights as to the average buying behaviours of your wearables
customers, important when answering you first set of questions.

Another limitation we encountered was discovered when tackling your
second set of questions regarding the racial complaints against Mingar's
devices. Since, as mentioned, Mingar does not collect any data on the
racial ethnicity of its users, we had to get creative with identifying
darker complexion customers from lighter ones. We utilized the emoji
modifier field populated by the customer. We must warn that some users
have no data within the emoji\_modifier field (labeled as NA) as they
have presumably not selected one, so we had to work within a subset of
the customer data. Moreover, it is not exactly accurate to suggest that
all customers use the same skin tone emoji modifier as their own,
because people have complex skin tones and may find it difficult to
select the right shade. Or they may simply slect the emoji that they
wish to express themselves with regardless of their own shades. However,
we believe, once again, that the average consumer will selected a fairly
similar emoji modifier to their own skin colour; and that was how we
were able to proceed with our analysis and provide some meaningful
results.

Not to toot our own horn, but Simplicity prides itself on being expert
model selectors, whereby we consider numerous models, six, in fact, per
question. Through an open and diverse lens, we consider various
predictors when answering both questions of focus to ensure we do not
exclude any important factors. Our strong foundational understanding of
distributions assisted us in deciding what the underlying distributions
of the datasets that we cleverly were able to wrangle, join together and
explore for the most meaningful models. Most importantly, our company's
strong core beliefs of accountability and integrity allows you to be
assured that the final model selection was based on results that were
authentic and reproducible with various illustrations, plots and tests
to showcase the data and results.

\newpage

\hypertarget{consultant-information}{%
\section{Consultant information}\label{consultant-information}}

\hypertarget{consultant-profiles}{%
\subsection{Consultant profiles}\label{consultant-profiles}}

\textbf{Aaisha Eid}. Aaisha is a senior consultant at Google. She
specializes in reproducible analysis and statistical communication.
Aaisha earned her Bachelor of Science, Specialist in Statistics Machine
Learning and Data Mining, from the University of Toronto in 2022.

\textbf{Charles Lu}. Charles is a junior consultant with Facebook. He
specializes in data wrangling and model selection. Charles earned their
Bachelor of Science, Specializing in Computer Science from the
University of Toronto in 2024.

\hypertarget{code-of-ethical-conduct}{%
\subsection{Code of ethical conduct}\label{code-of-ethical-conduct}}

Simplicity ensures that any information shared between itself and its
clients remain confidential and protected against sale for personal gain
or benefit. Conflicts of interest will surely be communicated to clients
transparently and accordingly. Personal information provided to by
clients will be treated anonymously and respectfully. Finally,
Simplicity is operated by proud statisticians that uphold the
professional statistical standards of procedure and analysis.

\newpage

\hypertarget{references}{%
\section{References}\label{references}}

R Core Team (2021). R: A language and environment for statistical
computing. R Foundation for Statistical Computing, Vienna, Austria. URL
\url{https://www.R-project.org/}.

Douglas Bates, Martin Maechler, Ben Bolker, Steve Walker (2015). Fitting
Linear Mixed-Effects Models Using lme4. Journal of Statistical Software,
67(1), 1-48. \url{doi:10.18637/jss.v067.i01}.

Wickham et al., (2019). Welcome to the tidyverse. Journal of Open Source
Software, 4(43), 1686, \url{https://doi.org/10.21105/joss.01686}

Achim Zeileis, Torsten Hothorn (2002). Diagnostic Checking in Regression
Relationships. R News 2(3), 7-10. URL
\url{https://CRAN.R-project.org/doc/Rnews/}

Hao Zhu (2021). kableExtra: Construct Complex Table with `kable' and
Pipe Syntax. R package version 1.3.4.
\url{https://CRAN.R-project.org/package=kableExtra}

David Robinson, Alex Hayes and Simon Couch (2021). broom: Convert
Statistical Objects into Tidy Tibbles. R package version 0.7.6.
\url{https://CRAN.R-project.org/package=broom}

Ben Bolker and David Robinson (2022). broom.mixed: Tidying Methods for
Mixed Models. R package version 0.2.9.3.
\url{https://CRAN.R-project.org/package=broom.mixed}

Chung-hong Chan, Geoffrey CH Chan, Thomas J. Leeper, and Jason Becker
(2021). rio: A Swiss-army knife for data file I/O. R package version
0.5.26.

Hadley Wickham and Evan Miller (2020). haven: Import and Export `SPSS',
`Stata' and `SAS' Files. R package version 2.3.1.
\url{https://CRAN.R-project.org/package=haven}

Hadley Wickham and Jim Hester (2020). readr: Read Rectangular Text Data.
R package version 1.4.0. \url{https://CRAN.R-project.org/package=readr}

Hadley Wickham (2021). rvest: Easily Harvest (Scrape) Web Pages. R
package version 1.0.2. \url{https://CRAN.R-project.org/package=rvest}

von Bergmann, J., Dmitry Shkolnik, and Aaron Jacobs (2021). cancensus: R
package to access, retrieve, and work with Canadian Census data and
geography. v0.4.2.

Unicode, Inc.~(2022). Unicode character table. Retrieved April 11, 2022,
from \url{https://unicode-table.com/en/}

\newpage

\hypertarget{appendix}{%
\section{Appendix}\label{appendix}}

\hypertarget{web-scraping-industry-data-on-fitness-tracker-devices}{%
\subsection{Web scraping industry data on fitness tracker
devices}\label{web-scraping-industry-data-on-fitness-tracker-devices}}

The industry data on fitness tracker devices was web scrapped in R from
\url{https://fitnesstrackerinfohub.netlify.app/}. the whole web scraping
process utilizes the R packages ``polite'' and ``rvest''. informative
user agents details including purpose of web scraping and contact
information was provided to website before scraping. It is important to
note that to respect the web scrapped contents, follow the terms and
services and Robots.txt of the website and only take what is needed.
Avoid web scraping at all if the target website provide public API to
access the data. Respect the crawl limit suggested by the website. The
web scraping industry data on fitness tracker devices are private data
saved under the folder raw-data, and will never be shared and published
on Github or any websites.

\hypertarget{accessing-census-data-on-median-household-income}{%
\subsection{Accessing Census data on median household
income}\label{accessing-census-data-on-median-household-income}}

The Canada Census data on median household income was accessed from
\url{https://censusmapper.ca/} through the APIs of R package
``cancencus'' and the private API key kindly provided by CensusMapper.
The data is governed by the Statistic Canada Open Data license and
Statistic Canada, and acknowledgment of them is required use the data.
The Census data are licensed data saved under the folder raw-data, and
will never be shared and published on Github or any websites. The API
key provided by CensusMapper should also not be shared anywhere as it is
a personal private key.

\hypertarget{accessing-postcode-conversion-files}{%
\subsection{Accessing postcode conversion
files}\label{accessing-postcode-conversion-files}}

The postal code conversion files was downloaded from
\url{https://mdl.library.utoronto.ca/collections/numeric-data/census-canada/postal-code-conversion-file}
through using University of Toronto student credentials. The file is
governed by Statistic Canada and stored in University of Toronto online
map and data library. Same as the Census data, the file is classified as
licensed data saved under the folder raw-data, and will never be shared
and published on Github or any websites.

\end{document}
